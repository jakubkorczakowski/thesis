\documentclass[a4paper,11pt, notitlepage, twoside, openany ]{report}
\usepackage[T1]{fontenc}
\usepackage[polish]{babel}
\usepackage[utf8]{inputenc}
\usepackage{lmodern}
\usepackage{enumitem}
\usepackage{indentfirst}
\usepackage{graphicx}
\usepackage{subcaption}
\usepackage{wrapfig}
\usepackage{fancyhdr}
\usepackage{lastpage}
\usepackage{listings}
\usepackage{spverbatim}
\usepackage{geometry}
\usepackage{amsmath,xparse}
\usepackage{listings}
\usepackage{color}
\usepackage{fix-cm}


\usepackage{url}
\usepackage{hyperref}

\pagestyle{fancy}
\fancyhf{}
\rfoot{Strona \thepage  \hspace{1pt} z \pageref{LastPage}}
% \lhead{}

\hypersetup{
	colorlinks=true,
	linkcolor=black,
	filecolor=magenta,      
	urlcolor=black,
}

\begin{document}
	
	\begin{titlepage}
		
		\newcommand{\HRule}{\rule{\linewidth}{0.5mm}} % Defines a new command for the horizontal lines, change thickness here
		
		\center % Center everything on the page
		
		\includegraphics[width = 28mm]{logo.jpg}\\[1.0cm] % Include a department/university logo - this will require the graphicx package

		%----------------------------------------------------------------------------------------
		%	HEADING SECTIONS
		%----------------------------------------------------------------------------------------
		
		\textsc{\LARGE Politechnika Warszawska}\\[0.5cm] % Name of your university/college
		\textsc{\Large Wydział Elektryczny }\\[1.0cm] % Major heading such as course name
		\textsc{ Instytut Sterowania i Elektroniki Przemysłowej \\
		Zakład Sterowania}\\[1.5cm] % Minor heading such as course title

		
		%----------------------------------------------------------------------------------------
		%	TITLE SECTION
		%----------------------------------------------------------------------------------------
		
		% \HRule \\[0.4cm]
		{ 	\fontsize{35}{30}\selectfont Praca dyplomowa inżynierska }\\[0.5cm] % Title of your document
		
		% \HRule \\[0.5cm]
		na kierunku \textsc{Informatyka} \\
		w specjalności \textsc{Inżynieria Danych i Multimediów}\\[1.0cm] 

		\LARGE Wykrywanie współwystępowania obiektów\\ na obrazach cyfrowych \\[2.0cm] 
		
		%----------------------------------------------------------------------------------------
		%	AUTHOR SECTION
		%----------------------------------------------------------------------------------------
		
		\begin{minipage}{0.4\textwidth}
			\begin{flushleft} \large
				\emph{Wykonał:}\\
				 % Your name
				\Large{Jakub \textsc{Korczakowski}} \\
				\textsc{\normalsize{nr albumu 291079}}
			\end{flushleft}
		\end{minipage}
		~
		\begin{minipage}{0.4\textwidth}
			\begin{flushright} \large
				\emph{Promotor:} \\
				dr inż. Grzegorz \textsc{Sarwas} \\[0.5cm] % Supervisor's
								
				\emph{Konsultant:} \\
				prof.  % Supervisor's 
			\end{flushright}
		\end{minipage}\\[2cm]
		
		% If you don't want a supervisor, uncomment the two lines below and remove the section above
		%\Large \emph{Author:}\\
		%John \textsc{Smith}\\[3cm] % Your name
		
		%----------------------------------------------------------------------------------------
		%	DATE SECTION
		%----------------------------------------------------------------------------------------
		
		{\large \today}\\[2cm] % Date, change the \today to a set date if you want to be precise
		
		%----------------------------------------------------------------------------------------
		%	LOGO SECTION
		%----------------------------------------------------------------------------------------
		
		
		%----------------------------------------------------------------------------------------
		
		\vfill % Fill the rest of the page with whitespace
		
	\end{titlepage}
	\tableofcontents
	\newpage



	\cite{cifar}

	\addcontentsline{toc}{chapter}{Bibliografia}
	\bibliographystyle{acm}
	\bibliography{./bibliografia} 
	
	
	
\end{document}
